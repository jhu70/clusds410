% Template for PLoS
% Version 3.5 March 2018
%
% % % % % % % % % % % % % % % % % % % % % %
%
% -- IMPORTANT NOTE
%
% This template contains comments intended
% to minimize problems and delays during our production
% process. Please follow the template instructions
% whenever possible.
%
% % % % % % % % % % % % % % % % % % % % % % %
%
% Once your paper is accepted for publication,
% PLEASE REMOVE ALL TRACKED CHANGES in this file
% and leave only the final text of your manuscript.
% PLOS recommends the use of latexdiff to track changes during review, as this will help to maintain a clean tex file.
% Visit https://www.ctan.org/pkg/latexdiff?lang=en for info or contact us at latex@plos.org.
%
%
% There are no restrictions on package use within the LaTeX files except that
% no packages listed in the template may be deleted.
%
% Please do not include colors or graphics in the text.
%
% The manuscript LaTeX source should be contained within a single file (do not use \input, \externaldocument, or similar commands).
%
% % % % % % % % % % % % % % % % % % % % % % %
%
% -- FIGURES AND TABLES
%
% Please include tables/figure captions directly after the paragraph where they are first cited in the text.
%
% DO NOT INCLUDE GRAPHICS IN YOUR MANUSCRIPT
% - Figures should be uploaded separately from your manuscript file.
% - Figures generated using LaTeX should be extracted and removed from the PDF before submission.
% - Figures containing multiple panels/subfigures must be combined into one image file before submission.
% For figure citations, please use "Fig" instead of "Figure".
% See http://journals.plos.org/plosone/s/figures for PLOS figure guidelines.
%
% Tables should be cell-based and may not contain:
% - spacing/line breaks within cells to alter layout or alignment
% - do not nest tabular environments (no tabular environments within tabular environments)
% - no graphics or colored text (cell background color/shading OK)
% See http://journals.plos.org/plosone/s/tables for table guidelines.
%
% For tables that exceed the width of the text column, use the adjustwidth environment as illustrated in the example table in text below.
%
% % % % % % % % % % % % % % % % % % % % % % % %
%
% -- EQUATIONS, MATH SYMBOLS, SUBSCRIPTS, AND SUPERSCRIPTS
%
% IMPORTANT
% Below are a few tips to help format your equations and other special characters according to our specifications. For more tips to help reduce the possibility of formatting errors during conversion, please see our LaTeX guidelines at http://journals.plos.org/plosone/s/latex
%
% For inline equations, please be sure to include all portions of an equation in the math environment.
%
% Do not include text that is not math in the math environment.
%
% Please add line breaks to long display equations when possible in order to fit size of the column.
%
% For inline equations, please do not include punctuation (commas, etc) within the math environment unless this is part of the equation.
%
% When adding superscript or subscripts outside of brackets/braces, please group using {}.
%
% Do not use \cal for caligraphic font.  Instead, use \mathcal{}
%
% % % % % % % % % % % % % % % % % % % % % % % %
%
% Please contact latex@plos.org with any questions.
%
% % % % % % % % % % % % % % % % % % % % % % % %

\documentclass[10pt,letterpaper]{article}
\usepackage[top=0.85in,left=2.75in,footskip=0.75in]{geometry}

% amsmath and amssymb packages, useful for mathematical formulas and symbols
\usepackage{amsmath,amssymb}

% Use adjustwidth environment to exceed column width (see example table in text)
\usepackage{changepage}

% Use Unicode characters when possible
\usepackage[utf8x]{inputenc}

% textcomp package and marvosym package for additional characters
\usepackage{textcomp,marvosym}

% cite package, to clean up citations in the main text. Do not remove.
% \usepackage{cite}

% Use nameref to cite supporting information files (see Supporting Information section for more info)
\usepackage{nameref,hyperref}

% line numbers
\usepackage[right]{lineno}

% ligatures disabled
\usepackage{microtype}
\DisableLigatures[f]{encoding = *, family = * }

% color can be used to apply background shading to table cells only
\usepackage[table]{xcolor}

% array package and thick rules for tables
\usepackage{array}

% create "+" rule type for thick vertical lines
\newcolumntype{+}{!{\vrule width 2pt}}

% create \thickcline for thick horizontal lines of variable length
\newlength\savedwidth
\newcommand\thickcline[1]{%
  \noalign{\global\savedwidth\arrayrulewidth\global\arrayrulewidth 2pt}%
  \cline{#1}%
  \noalign{\vskip\arrayrulewidth}%
  \noalign{\global\arrayrulewidth\savedwidth}%
}

% \thickhline command for thick horizontal lines that span the table
\newcommand\thickhline{\noalign{\global\savedwidth\arrayrulewidth\global\arrayrulewidth 2pt}%
\hline
\noalign{\global\arrayrulewidth\savedwidth}}


% Remove comment for double spacing
%\usepackage{setspace}
%\doublespacing

% Text layout
\raggedright
\setlength{\parindent}{0.5cm}
\textwidth 5.25in
\textheight 8.75in

% Bold the 'Figure #' in the caption and separate it from the title/caption with a period
% Captions will be left justified
\usepackage[aboveskip=1pt,labelfont=bf,labelsep=period,justification=raggedright,singlelinecheck=off]{caption}
\renewcommand{\figurename}{Fig}

% Use the PLoS provided BiBTeX style
% \bibliographystyle{plos2015}

% Remove brackets from numbering in List of References
\makeatletter
\renewcommand{\@biblabel}[1]{\quad#1.}
\makeatother



% Header and Footer with logo
\usepackage{lastpage,fancyhdr,graphicx}
\usepackage{epstopdf}
%\pagestyle{myheadings}
\pagestyle{fancy}
\fancyhf{}
%\setlength{\headheight}{27.023pt}
%\lhead{\includegraphics[width=2.0in]{PLOS-submission.eps}}
\rfoot{\thepage/\pageref{LastPage}}
\renewcommand{\headrulewidth}{0pt}
\renewcommand{\footrule}{\hrule height 2pt \vspace{2mm}}
\fancyheadoffset[L]{2.25in}
\fancyfootoffset[L]{2.25in}
\lfoot{\today}

%% Include all macros below

\newcommand{\lorem}{{\bf LOREM}}
\newcommand{\ipsum}{{\bf IPSUM}}





\usepackage{forarray}
\usepackage{xstring}
\newcommand{\getIndex}[2]{
  \ForEach{,}{\IfEq{#1}{\thislevelitem}{\number\thislevelcount\ExitForEach}{}}{#2}
}

\setcounter{secnumdepth}{0}

\newcommand{\getAff}[1]{
  \getIndex{#1}{Smith College}
}

\providecommand{\tightlist}{%
  \setlength{\itemsep}{0pt}\setlength{\parskip}{0pt}}

\begin{document}
\vspace*{0.2in}

% Title must be 250 characters or less.
\begin{flushleft}
{\Large
\textbf\newline{The Graphical Analysis on the Gap between Supply and Demand for Early Ed
Childcare Services} % Please use "sentence case" for title and headings (capitalize only the first word in a title (or heading), the first word in a subtitle (or subheading), and any proper nouns).
}
\newline
% Insert author names, affiliations and corresponding author email (do not include titles, positions, or degrees).
\\
Jocelyn Hu\textsuperscript{\getAff{Smith College}},
Kat Kyuchukova\textsuperscript{\getAff{Smith College}},
Paige Patrick\textsuperscript{\getAff{Smith College}}\\
\bigskip
\textbf{\getAff{Smith College}}SDS, 1 Chapin Way, Northampton, MA 01063\\
\bigskip
\end{flushleft}
% Please keep the abstract below 300 words
\section*{Abstract}
Community Labor United is a non-profit organization that is currently
working to investigate the negligence in childcare provision for low and
middle-income families within the Greater Boston Area, in order to
promote reform within the current system provided by the Department of
Early Education and Care. Using interactive maps, the research outlined
in this paper highlights two gaps in the current childcare structure.
The first being how the hours that early education childcare providers
keep do not support families that work outside the typical Monday
through Friday, 9:00am to 5:00pm job. The second being how the number of
slots for early education provision do not align with the number of
children age five and under that live within each neighborhood and
census tract. The interaction between these two gaps drastically impacts
the livelihood of working class families and it is our hope that this
research will provide evidence that the current system in place needs to
be reconstructed, in order to support all households in Massachusetts.

% Please keep the Author Summary between 150 and 200 words
% Use first person. PLOS ONE authors please skip this step.
% Author Summary not valid for PLOS ONE submissions.

\linenumbers

% Use "Eq" instead of "Equation" for equation citations.
\section{Introduction}\label{introduction}

\subsection{Community Labor United}\label{community-labor-united}

Community Labor United (CLU) is a non-profit organization, operating in
the Greater Boston Area, that works with other community-based
establishments and labor unions across Massachusetts to cultivate
strategic campaigns that protect and promote the interests of low and
middle-income working class families ({[}1{]}). Their overall goal is to
promote policies that advocate for accessible jobs, healthcare,
childcare, housing and environmental justice for working class
households ({[}1{]}). Through their Our Care That Works coalition that
launched publicly this year, Community Labor United aims to bring
together various local cooperative groups to confront the child care
crisis in Massachusetts. More specific to this research, CLU was
interested in examining the inconsistencies in childcare within the
Greater Boston Area, caused by the negligence of the Department of Early
Education and Care (EEC). This investigative exploration into childcare
provision gaps within the Greater Boston Area will allow Community Labor
United to communicate to the EEC the demand for the standardization of
childcare within Massachusetts.

\subsection{EEC Literature Review}\label{eec-literature-review}

The Department of Early Education and Care's mission is to maintain the
growth and development of all children, by providing quality childcare
programs and resources for families within their communities (EEC).
However, the research lead by Community Labor United and their
affiliated organizations shows that the EEC has been unable to
effectively execute their commitment for low and middle-income working
class families.\\
We will be adding more limitations after we receive some literature from
Sarah about the EEC and childcare!

\subsection{Research Question}\label{research-question}

As stated before, the general goal for this research was to examine the
inconsistencies in the EEC by highlighting childcare provision gaps in
the Greater Boston Area. More specifically, this project focused on
illustrating a disparity in operating hours and capacity for childcare
providers on the neighborhood level and census tract level. Because
Community Labor United is concentrated on understanding how the low and
middle-income households are impacted by the current childcare system,
our project focused on emphasizing the childcare demands needed for
working class families. Additionally, CLU was interested in
concentrating on early education provision care, which encompasses
childcare for children ages five and under. For hours, we looked at how
the non-typical work week is affected by the early education provision
that is currently supplied, since people with low and middle-incomes
work at times that operate outside the typical 9:00am - 5:00pm job. We
wanted to understand if there were a sufficient number of childcare
facilities that could offer early education provision outside Monday -
Friday, 7:30am - 6:00pm, for people in the Greater Boston Area.
Similarly for capacity, we wanted to highlight the lack of available
slots for early education provision for working households with children
age five and under. This is because working-class families typically
have all parents in the household in the workforce, so it is crucial
that the childcare capacity supply matches the demand.\\
Individually, the child provision gaps in hours and capacity shows how
the Department of Early Education and Care has neglected low and
middle-income families in specific ways. However, we also think it's
important to understand how the interaction between lack of available
childcare hours and lack of early education provision spots impact these
households overall. To best convey that interaction, we compiled our
visualizations and supporting information on an easily accessible
platform. This allows for the user to quickly navigate the various maps
for hours and capacity, and also share these findings with all
appropriate parties. It is our hope that Community Labor United will be
able to use the visualizations we've created to motivate the Department
of Early Education to make reforms to the current childcare system to be
more accommodating for all families in Massachusetts, regardless of
occupation or availability.

\section{Method}\label{method}

In order to quantify the gap in supply and demand for childcare in
Boston, we decided to focus on the capacity of the existing childcare
providers in comparison to the number of children who need childcare, as
well as the hours that providers are able to provide childcare in
comparison to the hours when childcare is most needed by those who work
outside standard hours. These areas were chosen because both for their
urgency in affecting access to and need for childcare, as well as for
the lack of existing research that exists on the gaps between the
childcare that exists and the childcare that is needed.

\subsection{Data}\label{data}

The data from our main analyses consisted of two data sources, one
addressing the supply side of the gaps in childcare, and the other
addressing the demand. The supply data source we used was given to us by
\emph{Community Labor United}, and it was a collection of providers from
the Department of Early Education and Care (EEC) in Massachusetts. The
initial dataset consisted of 8,318 observations, with each row
representing a childcare provider in Massachusetts. Since our project
focused specifically on the Boston area, we filtered the dataset to
contain only cities in Boston, which left us with 764 remaining
childcare providers. The cities included were Allston, Boston, South
Boston, Brighton, Charlestown, Dorchester, East Boston, Hyde Park,
Jamaica Plain, Mattapan, Roslindale, Roxbury, and West Roxbury. Cities
with the highest concentration of providers included central Boston
(100), Dorchester (203) and Roxbury (70).

There is no existing dataset that tells us about childcare demand
specifically, so to answer this question we used census data,
specifically the \emph{American Community Survey} (ACS), and filtered
for certain variables of interest. The following methodology will
discuss specific variables in detail, but data was derived from the 2016
American Community Survey and was filtered for Suffolk County,
Massachusetts.

We wanted to convey the gap in childcare by a geography that would be
big enough to generalize findings, but small enough that we could be
specific and not overlook any areas that might have findings of
interest. Through consultation with CLU, we decided on convey our
results by the neighborhood level. To do this, we accessed Boston
neighborhood shapefile data (Harvard World Map, 2019) and assigned each
EEC childcare provider to a certain neighborhood. Since the census data
of interest was only accessible by tract, we developed a file that
matched each census tract to a neighborhood, and included both tract
geometry and neighborhood geometry to allow for easy mapping. More about
this process is discussed in the challenges section.

\section{Capacity}\label{capacity}

\subsection{Supply}\label{supply}

To get a sense of the slots available for early education provision, we
took the EEC dataset and filtered only for providers that provide early
education childcare. We first used R and Python to clean the variables.
We then used the tidyverse package to group the providers by census
tract, and calculated the total number of slots for early education
childcare in each tract. Since we were interested in looking at
differences on both the tract level and the neighborhood level, we
repeated this process for neighborhood so that we also had the total
number of slots for early education childcare in each neighborhood. We
merged these datasets with the respective geometry for each geography,
to allow us to map the results.

\subsection{Demand}\label{demand}

Demand for capacity of childcare was assessed through the use of the
2016 ACS. The tigris and tidycensus packages were used in congruence
with an API key to access census data in R. We chose the ACS as opposed
to other forms of census data because it was the only survey that was
easily compatible with R that also had all data available for all
variables of interest on the tract level for Suffolk County.

To quantify the number of children that need childcare, we used the ACS
variable ``B23008'', which gives estimates per tract of the total number
of children under 6 years old, as well as the number of children under 6
in two-parent households with both parents in the labor force, as well
as single parent households (either mother or a father) with the parent
in the labor force. We then added up these three variables, with the
assumption that anybody with all parents in the labor force would need
childcare. To get a percentage of children under 6 with all parents in
the labor force, we divided this number by the total number of children
under 6. All calculations were per tract.

\subsection{Maps}\label{maps}

Three maps were made as a final product for visualizing the gap in
capacity. Two of these maps (Figure 1 and Figure 2) looked at supply and
demand at the tract level, mapping the raw number of children under 6
with all parents in the workforce as well as the raw number of slots
available for early ed care per tract, to be compared. The third map
(Figure 3) quantified the difference in supply and demand by
neighborhood, as a ratio of children under 6 with parents in the
workforce to available childcare slots. Since it is unrealistic that one
would restrict their childcare search to within their census tract, a
rather small boundary, we wanted to give a more realistic range of how
far a childcare search might go, hence the rationale for grouping by
neighborhood.

\section{Hours}\label{hours}

\subsection{Supply}\label{supply-1}

Our main focus with the question of hours was to get a sense of the
providers who provided childcare outside of the typical standard
workday, which we defined as anytime outside of 7:30am-6pm on weekdays.
We did not include weekday care since there were only a few providers
that provided any weekend care at all. Therefore to get our dataset for
supply, we created a variable based on the information about hours in
the EEC dataset to flag any provider that provided care during
nonstandard hours, and summarized the number of slots they had for those
hours.

\subsection{Demand}\label{demand-1}

One large problem with census data is that in any workday related
variables, it assumes workers work 5 days a week, with the same number
of hours every day. This definition contradicts the purpose of our
study, which is to investigate those who work nonstandard hours.
Therefore, the only census variable that made sense to use when
investigating hours was the time leaving work. Using previous ACS
methodology given to us by CLU, we split the nonstandard times leaving
for work into three categories: early mornings if they leave for work
anytime between 12am-6:29am, evenings if they leave for work anytime
between 11am-3:59pm, late evening/overnight if they leave for work
anytime between 4pm-11:59pm. We took the ACS variables corresponding to
these responses in variable B08302, and divided the number of people in
each category by the total number of people in the workforce to get a
percentage of people in each category. We also summarized the number of
people in the three nonstandard time chunks to get a total percentage of
people leaving during any nonstandard hour.

\subsection{Maps}\label{maps-1}

Five maps were made to visualize the gap in supply and demand for hours.
Three of the maps were choropleth maps by tract visualizing the
percentage of people who left at each of the nonstandard times: early,
evening, and late evening/overnight (See Figure 1 for an example of this
map). Another similar map was made by tract using the aggregate
percentage of all people leaving during any of these nonstandard hours.
Finally, to visualize supply, a map was made of the number of slots
available for nonstandard hours of childcare by provider by tract.

\section{Results and Discussion}\label{results-and-discussion}

This section summarizes the data utilized and how the maps we create
illustrate the gap between demand and supply of childcare services for
children under 6 in the Boston area. We start by presenting the summary
statistics of the data sets, analyzing each individual map, and then
proceed to analyze the relationship among them.

\subsection{Summary Statistics}\label{summary-statistics}

\begin{verbatim}
## 
## Descriptive table(early ed only)
## ======================================================================
## Statistic                                     Mean  St. Dev. Min  Max 
## ----------------------------------------------------------------------
## Average number of slots by providers (slot)    25      36     3   325 
## Average number of slots by neighborhood(slot) 1,669  1,064    6  3,430
## Average number of slots by tract(slot)         210    200     6  1,014
## ----------------------------------------------------------------------
\end{verbatim}

The table in Figure 1 presents the summary statistics on the
distribution of the capacities of all providers and capacities available
in neighborhoods and tracts. The mean capacity of providers is 24.7, of
all neighborhoods in Boston is 1469, and of tracts in Boston is 169.2.

\begin{verbatim}
## 
## Descriptive table(weekdays off hour services for early ed)
## ======================================================================
## Statistic                                        Mean St. Dev. Min Max
## ----------------------------------------------------------------------
## Average number of off-hour slots by providers     26     35     6  140
## Average number of off-hour slots by neighborhood  41     40     8  140
## Average number of off-hour slots by tract         26     35     6  140
## ----------------------------------------------------------------------
\end{verbatim}

\subsection{Graphs for Capacity}\label{graphs-for-capacity}

In this section we conduct graphical analysis on the two variables we
are interested in: capacity and hours.

\begin{figure}

{\centering \includegraphics[width=0.25\linewidth]{fig1} 

}

\caption{Figure1: Number of children under 6 from working families}\label{fig:unnamed-chunk-3}
\end{figure}

Figure 1 shows the number of children under 6 whose parents are in the
labor force (i.e., actively looking for jobs) on the level of
neighborhoods. We are especially in them since they are target consumers
of childcare services for early ed. Being in the labor force means that
the parents might to too busy to pay enough attention and provide enough
cares to their children, thus requiring the aids from childcare
providers. According to the map, tracts that have the largest number of
children under 6 who would need childcare services are mainly located in
the east and west parts of Boston (i.e., upper and lower regions on the
map), including East Boston, West Roxbury, Hyde Park, and Dorchester.

\begin{figure}

{\centering \includegraphics[width=0.25\linewidth]{fig2} 

}

\caption{Figure2: Number of slots available for early ed by tracts}\label{fig:unnamed-chunk-4}
\end{figure}

Figures 2 shows the number of childcare slots available for children
under 6 on the level of tracts. The two tracts that contain more slots
than others are located in Dorchester and Downtown, but compared to
Figure 2, the slots are apparently more evenly distributed across
tracts. It is worth paying attention to that providers in the regions
where there is a large number of children under 6, including the east
and west parts of Boston, do not provide enough number of slots to feed
the need of families in these areas. The difference between supply and
demand explains the high desert ratio in these regions as shown in
Figure 3.

\begin{figure}

{\centering \includegraphics[width=1\linewidth]{fig3} 

}

\caption{Figure3: Gap between supply and demand of capacities}\label{fig:unnamed-chunk-5}
\end{figure}

We first visualize the gap between supply and demand for childcare
services for early education through the capacity variables. Figure 1
shows the ratio of children under age 5 to the cumulative child care
capacity in the neighborhoods of Boston. The darker the red color, the
more children are competing to obtain a licensed child care slot, and
therefore the harder it is for a child to obtain the childcare service
he/she needs. According to the graph, East Boston, Charlestown, South
Boston, Dorchester, Mattapan, Roslindale, Hyde Park, and West Roxbury
could be classified as child care deserts. To solve this issue, more
childcare providers should be established in those neighborhoods and
current providers should expand their capacities. To get a closer look
at the capacity, the two figures below show the demand and supply of
capacities by tracts.

\subsection{Graphs for Hours}\label{graphs-for-hours}

In this section, we look into the working hours of families and the
operating hours of providers to examine whether there is a mismatch.

\begin{figure}

{\centering \includegraphics[width=0.25\linewidth]{fig4} 

}

\caption{Figure4: Percentage maps by tracts on people departure time for work in the morning}\label{fig:unnamed-chunk-6}
\end{figure}

Figure 4 shows the percentage of parents in tracts who depart for jobs
early in the morning. On average, people who live in the East Boston and
West Boston areas (north and south areas on the map) leave earlier for
works than people who live in other regions. These are also the regions
where there are relatively larger number of children under 6 in need of
childcare services. The tracts where the largest percentages of people
leave for jobs early are located at are waterfront areas of South Boston
and Dorchester.

\begin{figure}

{\centering \includegraphics[width=0.25\linewidth]{fig5} 

}

\caption{Figure5: Percentage maps by tracts on people departure time for home in the evening}\label{fig:unnamed-chunk-7}
\end{figure}

Figure 5 shows the percentage of parents in tracts who leave for home in
the late evening. On average, people who live in the East Boston, West
Boston areas (north and south areas on the map), and waterfront tracts
Brighton and Allston leave from workplaces for home later than people
who live in other regions. The tracts where the largest percentages of
people leave for home later than usual are located at Roxbury and the
waterfront areas of Jamaica Plain.

\begin{figure}

{\centering \includegraphics[width=0.25\linewidth]{fig6} 

}

\caption{Percentage maps by tracts of people whose departure time for home is overnight}\label{fig:unnamed-chunk-8}
\end{figure}

Figure 6 shows the percentage of parents in tracts who depart for home
from the workplace overnight. On average, people who live in South
Boston and West Boston areas (north and south areas on the map) leave
later for home than people who live in other regions. These are also the
regions where larger percentage of people leave early and where there
are relatively larger number of children under 6 in need of childcare
services. The tracts where the largest percentages of people leave for
jobs early are located at are waterfront areas of South Boston and
Dorchester.

\begin{figure}

{\centering \includegraphics[width=0.25\linewidth]{fig7} 

}

\caption{Number of slots in certain tracts by providers who operate off hour during weekdays for early ed}\label{fig:unnamed-chunk-9}
\end{figure}

Figure 7 shows the number of slots available for early by providers that
open before 7:30am and close after 6pm during weekdays. It indicates
that there is a significant lack of capacities in regions where families
have a difficult commute time. In general, people from working families
living in East Boston, West Boston, South Boston, and the waterfront
areas of Allston, Brighton, and Jamaica Plain are most in need of off
hour childcare services. However, we could see that there are only a few
slots offered by childcare providers that operate off hour available for
children in West Boston and South Boston, and none for children in South
Boston and the waterfront areas of Allston, Brighton, and Jamaica Plain.
Accordingly, more off hour slots should be provided in these regions.

\section{Challenges and Limitations}\label{challenges-and-limitations}

\subsection{Challenges}\label{challenges}

There are two main challenges that we deal with for this project. First
is data cleaning for the EEC dataset. The EEC dataset has a messy data
format and a lot of columns that contain information on several
different variables. To extract information that we need to create the
capacity and hours variables, we utilize various functions and packages
from R and Python to clean, merge, and spread variables from the
original dataset, such as minimum age, rates, open and close time of
providers.

Second is the geographical classification. One of the narratives of this
project is to look at the supply and demand by the tract and
neighborhood levels, the information of which are not provided in the
original datasets we are given. We deal with this issue by geocoding the
providers' locations, deciding which polygon each provider falls within
using the over function from the rgeos package.

\subsection{Limitations}\label{limitations}

First some of the variables from census datasets are based on
assumptions different from real life situations. Specifically, people's
departure time for work is created based on the assumption that people
work for the same amount of hours everyday while the parents we are most
interested in are those that have irregular working hours.

Second, our general lack of ability to match census data variables with
variables EEC provide to compare different variables for the same
construct. Although we successfully create the desert ratio by using the
capacity and children population from the EEC and census datasets, there
are many other variables the project could look into, such as income
level of citizens vs.~subsidy of providers.

Third, there is a lack of data in the EEC dataset about important
variables, such as the capacity of each providers by age groups and
subsidy amount in dollars. The absence of these variables prevents us
from looking more deeply into the supply for childcare services. There
are many variables that we clean but do not look into due to the limited
amount of time we have, such as subsidy, availability of drop-in and
emergency services.

\section{Discussion and Conclusion}\label{discussion-and-conclusion}

To examine the inconsistencies in the EEC by highlighting childcare
provision gaps in the Greater Boston Area, we utilize EEC datasets and
census datasets to create maps that illustrate a disparity in operating
hours and capacity for childcare providers on the neighborhood level and
census tract level. In a word, East Boston and West Boston
neighborhoods, including Allston, Boston, South Boston, Brighton,
Charlestown, Dorchester, East Boston, Hyde Park, Jamaica Plain,
Mattapan, Roslindale, Roxbury, and West Roxbury, are the regions where
the gaps apparently exist. According to the visualizations of the
``capacity'' variable, providers in East Boston and West Boston, where
there is a large number of children under 6, do not offer enough number
of slots to feed the need of families in these areas. Similarly, the
maps on the ``hours'' variable indicate that there is a lack of
childcare off hour services for working families living in East Boston,
West Boston, South Boston, and the waterfront areas of Allston,
Brighton, and Jamaica Plain are most in need of off hour childcare
services. According to the median income level map of Boston, most of
these neighborhoods are also regions where income levels are lower than
the average income level of the Boston area ({[}2{]}). The limited
financial capability of citizens of these areas might force them to work
during irregular hours and to send their children to the childcare
providers. On the other hand, the providers in these regions could not
provide enough slots for children under 6 as they could not earn as much
money from these low-income families as they could from families in the
central areas. To resolve the gap, the policy intervention and support
from EEC could be urgent.\\
In a word, our graphical analysis successfully illustrates the existence
of the gap between supply and demand for early ed childcare services in
the Boston area. For researchers who are interested in digging more
deeply into this topic in the future, we suggest several directions they
could explore into. First, people could match more variables on the
demand for childcare services with variables on the supply for childcare
services to illustrate the existence of gaps. Second, people could try
to obtain a more comprehensive dataset from EEC to have a closer look at
the supply of childcare services for different age groups. Third, people
could conduct a childcare survey by themselves to have first-hand
information directly from the providers and families in need of
childcare services in Boston. Fourth, people could encharge the studied
area to the whole mass

\section*{References}\label{references.unumbered}
\addcontentsline{toc}{section}{References}

\hypertarget{refs}{}
\hypertarget{ref-CLU}{}
1. Admin. Community labor united -- empowering community and labor
organizations that protect and promote the interests of working
families. {[}Internet{]}. Community Labor United. Community Labor
United; 2019. Available: \url{http://massclu.org/}

\hypertarget{ref-bostonmap}{}
2. Boston income map {[}Internet{]}. City-data. Available:
\url{http://www.city-data.com/income/income-Boston-Massachusetts.html}

\hypertarget{ref-bostonshapefile}{}
3. Harvard worldmap {[}Internet{]}. WorldMap. Available:
\url{http://worldmap.harvard.edu/data/geonode:boston_neighborhood_shapefiles_iq5}

\hypertarget{ref-R-base}{}
4. R Core Team. R: A language and environment for statistical computing
{[}Internet{]}. Vienna, Austria; 2017. Available:
\url{https://www.R-project.org/}

\hypertarget{ref-R-papaja}{}
5. Aust F, Barth M. papaja: Create APA manuscripts with R Markdown
{[}Internet{]}. 2018. Available: \url{https://github.com/crsh/papaja}

\hypertarget{ref-R-dplyr}{}
6. Wickham H, François R, Henry L, Müller K. Dplyr: A grammar of data
manipulation {[}Internet{]}. 2019. Available:
\url{https://CRAN.R-project.org/package=dplyr}

\hypertarget{ref-R-forcats}{}
7. Wickham H. Forcats: Tools for working with categorical variables
(factors) {[}Internet{]}. 2018. Available:
\url{https://CRAN.R-project.org/package=forcats}

\hypertarget{ref-R-ggformula}{}
8. Kaplan D, Pruim R. Ggformula: Formula interface to the grammar of
graphics {[}Internet{]}. 2017. Available:
\url{https://CRAN.R-project.org/package=ggformula}

\hypertarget{ref-R-ggplot2}{}
9. Wickham H. Ggplot2: Elegant graphics for data analysis
{[}Internet{]}. Springer-Verlag New York; 2016. Available:
\url{https://ggplot2.tidyverse.org}

\hypertarget{ref-R-lattice}{}
10. Sarkar D. Lattice: Multivariate data visualization with r
{[}Internet{]}. New York: Springer; 2008. Available:
\url{http://lmdvr.r-forge.r-project.org}

\hypertarget{ref-R-leaflet}{}
11. Cheng J, Karambelkar B, Xie Y. Leaflet: Create interactive web maps
with the javascript 'leaflet' library {[}Internet{]}. 2018. Available:
\url{https://CRAN.R-project.org/package=leaflet}

\hypertarget{ref-R-leaflet.extras}{}
12. Karambelkar B, Schloerke B. Leaflet.extras: Extra functionality for
'leaflet' package {[}Internet{]}. 2018. Available:
\url{https://CRAN.R-project.org/package=leaflet.extras}

\hypertarget{ref-R-mapview}{}
13. Appelhans T, Detsch F, Reudenbach C, Woellauer S. Mapview:
Interactive viewing of spatial data in r {[}Internet{]}. 2018.
Available: \url{https://CRAN.R-project.org/package=mapview}

\hypertarget{ref-R-Matrix}{}
14. Bates D, Maechler M. Matrix: Sparse and dense matrix classes and
methods {[}Internet{]}. 2017. Available:
\url{https://CRAN.R-project.org/package=Matrix}

\hypertarget{ref-R-mosaic}{}
15. Pruim R, Kaplan DT, Horton NJ. The mosaic package: Helping students
to 'think with data' using r. The R Journal. 2017;9: 77--102. Available:
\url{https://journal.r-project.org/archive/2017/RJ-2017-024/index.html}

\hypertarget{ref-R-mosaicData}{}
16. Pruim R, Kaplan D, Horton N. MosaicData: Project mosaic data sets
{[}Internet{]}. 2016. Available:
\url{https://CRAN.R-project.org/package=mosaicData}

\hypertarget{ref-R-purrr}{}
17. Henry L, Wickham H. Purrr: Functional programming tools
{[}Internet{]}. 2019. Available:
\url{https://CRAN.R-project.org/package=purrr}

\hypertarget{ref-R-readr}{}
18. Wickham H, Hester J, Francois R. Readr: Read rectangular text data
{[}Internet{]}. 2017. Available:
\url{https://CRAN.R-project.org/package=readr}

\hypertarget{ref-R-sf}{}
19. Pebesma E. Simple Features for R: Standardized Support for Spatial
Vector Data. The R Journal. 2018; Available:
\url{https://journal.r-project.org/archive/2018/RJ-2018-009/index.html}

\hypertarget{ref-R-stringr}{}
20. Wickham H. Stringr: Simple, consistent wrappers for common string
operations {[}Internet{]}. 2019. Available:
\url{https://CRAN.R-project.org/package=stringr}

\hypertarget{ref-R-tibble}{}
21. Müller K, Wickham H. Tibble: Simple data frames {[}Internet{]}.
2019. Available: \url{https://CRAN.R-project.org/package=tibble}

\hypertarget{ref-R-tidycensus}{}
22. Walker K. Tidycensus: Load us census boundary and attribute data as
'tidyverse' and 'sf'-ready data frames {[}Internet{]}. 2019. Available:
\url{https://CRAN.R-project.org/package=tidycensus}

\hypertarget{ref-R-tidyr}{}
23. Wickham H, Henry L. Tidyr: Easily tidy data with 'spread()' and
'gather()' functions {[}Internet{]}. 2019. Available:
\url{https://CRAN.R-project.org/package=tidyr}

\hypertarget{ref-R-tidyverse}{}
24. Wickham H. Tidyverse: Easily install and load the 'tidyverse'
{[}Internet{]}. 2017. Available:
\url{https://CRAN.R-project.org/package=tidyverse}

\hypertarget{ref-R-tigris}{}
25. Walker K. Tigris: Load census tiger/line shapefiles {[}Internet{]}.
2018. Available: \url{https://CRAN.R-project.org/package=tigris}

\hypertarget{ref-EEC}{}
26. Weber TL. Department of early education and care {[}Internet{]}.
Mass.gov. Common Wealth of Massachusetts; 2019. Available:
\url{https://www.mass.gov/orgs/department-of-early-education-and-care}

\nolinenumbers


\end{document}

